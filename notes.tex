\documentclass{article}

\usepackage[a4paper, total={6in, 8in}]{geometry} % text area 6in x 8in
\usepackage{amssymb}
\usepackage{amsmath}
\usepackage{setspace}

\setlength\parindent{0pt}

\newcommand{\N}{\mathbb{N}}
\newcommand{\Z}{\mathbb{Z}}
\newcommand{\R}{\mathbb{R}}
\newtheorem{example}{Example}[section]
\newtheorem{notes}{Notes}[example]

\title{CSC236, Francois Pitt \\ Introduction to the Theory of Computation}
\author{}
\date{Fall 2023}

\begin{document}
    
\maketitle

\tableofcontents

\pagebreak

\section{Induction}

Example proof of QRT:

Theorem: $\forall m, \forall n > 0, \exists q, \exists r < n, m = qn + r $

Let $m_0 \in \N, n_0 \in \N$ and $n_0 > 0$.

Remark: we will apply well-ordering principle. 

Consider $R=\{r \in \N : \exists q, m_0 = qn_0 + r\}$

Claim: $R \neq \emptyset$ to apply WOP\@.

Justification: $m_0 \in \R \implies m_0 = 0\cdot n_0 + m_0$ so $R$ always contains at least one element.

Thus, $R$ is a set of natural numbers and is non-empty. By WOP, $R$ contains a smallest element $r_0$.

Thus, $r_0 \in \R \implies \exists q_0, m_0 = q_0n_0 + r_0$.

And $r_0$ is the smallest element of $R$, i.e. $\forall r \in R, r_0 \leq r$.

We want to show $r_0 < n_0$.

Suppose $r_0 \geq n_0$.

Then by how $r_0$ was picked, $r_0$ is the smallest $r \in \R$. However, if $r_0 \geq n_0$, then there would be a smaller remainder (precisely $r_0 - n_0$) that would contradict $r_0$ being the smallest element.

In essence:

$m_0 = q_0 n_0 + r_0$

$m_0 = (q_0 + 1)n_0 + (r_0-n_0)$ if $r_0 \geq n_0$.

Contradicting $r_0$ being the smallest element of $R$.

Or more formally, $r_0 \leq r_0 - n_0$ and $r_0 > r_0 - n_0$ are simultaneously true, which is a contradiction.

Thus $r_0 < n_0$.

Exercise:

Plug in numbers.

\hrulefill

Claim:

Any statement that can be proved using simple induction or complete induction or well-ordering can be proved using any one of the principles.

How?

Show PSI $\iff$ PCI $\iff$ WOP $\iff$ PSI \dots

Show
\begin{enumerate}
    \item PSI $\implies$ PCI
    \item PCI $\implies$ WOP
    \item WOP $\implies$ PSI
\end{enumerate}

1. Assume PSI holds for all predicates. We want to prove PCI.

Let $P: \N \to \{T, F\}$ be a predicate.

Assume $\forall n, (\forall k < n P(k)) \implies P(n)$.

WTS: $\forall n, P(n)$.

omitted due to not paying attn in lecture

\pagebreak

\section{Algorithm Analysis}
\onehalfspacing

Algorithm analysis (AA) is the field of analyzing algorithms. Wow. We can analyze:

\begin{itemize}
    \item Correctness
    \item Runtime (time complexity)
    \item Memory (space complexity)
    \item Communication complexity (e.g. distributed services)
    \item and more...
\end{itemize}

Runtime is the main focus for now.

Runtime measures basic operations and is expressed as a function of the input size.

However, this is under the assumption that the function behaves predictably for inputs. If not, we can study worst-case/best-case/average-case runtimes.

Runtime uses asymptotic notation ($\textbf{O}, \Omega, \Theta)$ to prove bounds.

\subsection{Algorithm Correctness}

\textbf{Idea:} Instead of designing an algorithm and proving its correctness, design an algorithm from a correct proof.

We can do this with:
\begin{itemize}
    \item Preconditions
    \begin{itemize}
        \item Intuitively, describes "valid" inputs
        \item Formally, a precondition is something we assume to be true before running the code
        \item This can be abused with impractically strong preconditions. Thus, we want weak preconditions for minimal constraints.
    \end{itemize}
    \item Postconditions
    \begin{itemize}
        \item Intuitively, describes "expected" outputs
        \item Formally, a postcondition is something we assume to be true after running the code
        \item In practice, we want to maximize the relevant details we know about the postconditions. Thus, we want strong postconditions.
    \end{itemize}
\end{itemize}

We see that changing the pre/post conditions for any algorithm changes the correctness of said algorithm.

Thus, we can formally define correctness as:

For every input, preconditions hold $\implies$ code terminates and postconditions hold.

\begin{example}
    Consider the coin denomination proof.
\end{example}

We define a function $\texttt{Change(n)}$ with preconditions: $n \in \N, n \geq 12$ and postconditions: $\texttt{Change(n)} = (a, b)$ where $a$ is the number of 3-cent coins and $b$ is the number of 5-cent coins.

\begin{verbatim}
    Change(n):
        if n == 12: return (4, 0)
        if n == 13: return (2, 1)
        if n == 14: return (0, 2)
        if n >= 15: 
            (a, b) = Change(n - 3)
            return (a + 1, b)
\end{verbatim}

We can now state correctness. For each $n \in \N, n \geq 12 \implies \texttt{Change(n)}$ terminates and returns $\texttt{(a, b)}$ such that $n = 3a + 7b$.

\begin{notes}
\end{notes}
\begin{enumerate}
    \item For recursive algorithms, prove by induction on input size $n$, usually with complete induction.
    \item Must justify that the recursive call is made on a smaller input that satisfies precondtions.
\end{enumerate}
\end{document}